En esta sección, a manera de orientación, en general se deben incluir:
\begin{itemize}
\item Contextualización del problema 
\item Descripción precisa del problema
\item Justificación del estudio
\item Marco Teórico o Conceptual
\item Delimitación del problema
\item Relevancia del estudio
\end{itemize}



\subsection{Contextualización del problema}
Situar el problema de investigación dentro de un contexto amplio,
proporcionando información relevante sobre el tema en cuestión. Esto
implica explicar por qué el problema es relevante, cuál es su origen y
su impacto en el campo de estudio y en la sociedad, si aplica. Además,
es importante hacer referencia a estudios previos y avances en la
temática.


\subsection{Descripción precisa del problema}
Debe ser claramente identificado y definido. Esto incluye una exposición clara sobre qué se va a investigar, cuáles son las preguntas clave y cómo el problema se articula dentro del marco teórico existente. La descripción debe ser lo suficientemente precisa como para evitar ambigüedades, dejando en claro qué es lo que se quiere solucionar o investigar.

\subsection{Justificación del estudio}
En esta parte se explica por qué es importante investigar este problema, y se argumenta por qué es necesario abordar el tema desde la perspectiva propuesta. La justificación puede incluir la relevancia científica, práctica o social del estudio, y también puede aludir a gaps de conocimiento en la literatura actual.

\subsection{ Marco teórico o conceptual }
Bajo ciertas circunstancias el planteamiento del problema también
puede incluir una breve referencia al marco teórico o conceptual, es
decir, los conceptos clave, teorías y modelos que se usarán para
abordar el problema de investigación. Aquí, se delinean las
perspectivas y enfoques que se emplearán para analizar y comprender el
problema.

\subsection{Delimitación del problema}
Es crucial precisar los límites del estudio: ¿qué aspectos del
problema se investigarán y cuáles no? Esto ayuda a evitar una
sobrecarga de variables y a enfocar la investigación en aspectos
específicos.

\subsection{Relevancia del estudio}
Explicar cómo los resultados de la investigación contribuirán al
avance del conocimiento en el área de estudio, a la solución de un
problema práctico o a la mejora de políticas, si es el caso.
